%%%%%%%%%%%%%%%%%
% This is an sample CV template created using altacv.cls
% (v1.4, 12 Apr 2021) written by LianTze Lim (liantze@gmail.com). Now compiles with pdfLaTeX, XeLaTeX and LuaLaTeX.
%
%% It may be distributed and/or modified under the
%% conditions of the LaTeX Project Public License, either version 1.3
%% of this license or (at your option) any later version.
%% The latest version of this license is in
%%    http://www.latex-project.org/lppl.txt
%% and version 1.3 or later is part of all distributions of LaTeX
%% version 2003/12/01 or later.
%%%%%%%%%%%%%%%%

%% Use the "normalphoto" option if you want a normal photo instead of cropped to a circle
% \documentclass[10pt,a4paper,normalphoto]{altacv}

\documentclass[10pt,a4paper,ragged2e,withhyper]{altacv}
%% AltaCV uses the fontawesome5 and packages.
%% See http://texdoc.net/pkg/fontawesome5 for full list of symbols.

% Change the page layout if you need to
\geometry{left=1.25cm,right=1.25cm,top=1.5cm,bottom=1.5cm,columnsep=1.2cm}

% The paracol package lets you typeset columns of text in parallel
\usepackage{paracol}

% Change the font if you want to, depending on whether
% you're using pdflatex or xelatex/lualatex
\ifxetexorluatex
  % If using xelatex or lualatex:
  \setmainfont{Roboto Slab}
  \setsansfont{Lato}
  \renewcommand{\familydefault}{\sfdefault}
\else
  % If using pdflatex:
  \usepackage[rm]{roboto}
  \usepackage[defaultsans]{lato}
  % \usepackage{sourcesanspro}
  \renewcommand{\familydefault}{\sfdefault}
\fi

% Change the colours if you want to
\definecolor{SlateGrey}{HTML}{2E2E2E}
\definecolor{LightGrey}{HTML}{666666}
\definecolor{DarkPastelRed}{HTML}{450808}
\definecolor{DarkPastelGreen}{HTML}{669D6D}
\definecolor{PastelRed}{HTML}{8F0D0D}
\definecolor{PastelGreen}{HTML}{79A054}
\definecolor{GoldenEarth}{HTML}{E7D192}
\colorlet{name}{black}
\colorlet{tagline}{PastelGreen}
\colorlet{heading}{DarkPastelGreen}
\colorlet{headingrule}{GoldenEarth}
\colorlet{subheading}{PastelGreen}
\colorlet{accent}{PastelGreen}
\colorlet{emphasis}{SlateGrey}
\colorlet{body}{LightGrey}

% Change some fonts, if necessary
\renewcommand{\namefont}{\Huge\rmfamily\bfseries}
\renewcommand{\personalinfofont}{\footnotesize}
\renewcommand{\cvsectionfont}{\LARGE\rmfamily\bfseries}
\renewcommand{\cvsubsectionfont}{\large\bfseries}


% Change the bullets for itemize and rating marker
% for \cvskill if you want to
\renewcommand{\itemmarker}{{\small\textbullet}}
\renewcommand{\ratingmarker}{\faCircle}

%% Use (and optionally edit if necessary) this .cfg if you
%% want to use an author-year reference style like APA(6)
%% for your publication list
\input{pubs-authoryear.cfg}

%% Use (and optionally edit if necessary) this .cfg if you
%% want an originally numerical reference style like IEEE
%% for your publication list
% \input{pubs-num.cfg}

%% sample.bib contains your publications
\addbibresource{sample.bib}

\begin{document}
\name{Jean-Christophe Legatte}
\tagline{Ingénieur Réseaux}
%% You can add multiple photos on the left or right
\photoR{2.8cm}{Jc_Legatte.jpeg}
% \photoL{2.5cm}{Yacht_High,Suitcase_High}

\personalinfo{%
  % Not all of these are required!
  \email{jclegatte@gmail.com}
  \phone{+336 01 79 26 25}
  \location{Maisons-Laffite, France}
  \linkedin{https://www.linkedin.com/in/jean-christophe-legatte-b9b86298/}
  \github{https://github.com/jclegatte}
  %% You can add your own arbitrary detail with
  %% \printinfo{symbol}{detail}[optional hyperlink prefix]
  % \printinfo{\faPaw}{Hey ho!}[https://example.com/]
  %% Or you can declare your own field with
  %% \NewInfoFiled{fieldname}{symbol}[optional hyperlink prefix] and use it:
  % \NewInfoField{gitlab}{\faGitlab}[https://gitlab.com/]
  % \gitlab{your_id}
}

\makecvheader
%% Depending on your tastes, you may want to make fonts of itemize environments slightly smaller
% \AtBeginEnvironment{itemize}{\small}

%% Set the left/right column width ratio to 6:4.
\columnratio{0.6}

% Start a 2-column paracol. Both the left and right columns will automatically
% break across pages if things get too long.
\begin{paracol}{2}
\cvsection{Expériences professionnels}
\cvevent{}{BLADE - SHADOW / Cloud Computing}{Depuis Sept 2020 }{Paris}
\begin{itemize}
\item Exploitation de 8 Datacenters internationnaux et 2 PoPs
\item Exploitation des ASNs publiques (RIPE, ARIN, APNIC)
\item Capacitaire de 2.5+Tb/s en transit/ix/peering sur plus de 40 peers
\item Deploiement de DCs multi-vendeurs avec outils d'automatisation fait maison. BUILD/RUN/Astreintes
\item BGP EVPN/VXLAN et BGP To The Host sur architecture Clos
\item Gestion de projet téchnique : Migration Nos Cumulus NOS, automatisation BGPtth Juniper, déploiement NFTables, POC Sonic
\end{itemize}
\cvevent{}{ORANGE PORTAIL - Hébergement/Exploitation}{Depuis Aoùt 2010 à Sept 2020}{Bagnolet}
\begin{itemize}
\item TechLead datacenter : Déploiement et exploitation de l'infrastructure réseaux pour fournir les services Orange Portail (www.orange.fr) et création du cloud privé Orange Portail Digital Factory
\item 3DCs, 800 équipements multi-vendeurs. BUILD/RUN/Astreintes
\item Firewalls, routeurs et repartiteurs de charge L4/L7 (Appliance et OpenSource)
\item Provisioning, automatisation et CI/CD. Config management et supervision - Ansible et API
\item Métriques du réseaux avec SNMP, SFLOW et GRPC
\item Gestion de projet téchnique : Q-Fabric Juniper 10-40Gbits / Coeur de réseaux Cisco Nexus 9000 et Juniper McLag / Clos L3 fabric (OSPF et Mellanox-Cumulus) / IPV6 - Dual Stack pour firewalls et Load balancer / Juniper CoS pour plateforme QFX
\end{itemize}
\cvevent{}{Société Générale - ALD AUTOMOTIVE HOLDING}{de sept 2007 à âout 2010}{Clichy}
\begin{itemize}
\item Responsable Réseaux Holding International (40 pays), BUILD et RUN pour LAN, MAN, WAN
\item Design d'architecture réseau (routage dynamique, optimisation Qos, sécurité)
\item Connectivité internationnal et résilience sur 2 DCs avec MPLS / BGP / VPN IPSec
\item SLA réseaux, budget et gestion des relations fournisseurs
\item Gestion de projet technique : déploiment du PRA sur un nouveau Datacenter
\end{itemize}

% use ONLY \newpage if you want to force a page break for
%%% ONLY the current column
%%\newpage
%%
%%\cvsection{Publications}
%%
%%\nocite{*}
%%
%%\printbibliography[heading=pubtype,title={\printinfo{\faBook}{Books}},type=book]
%%
%%\divider
%%
%%\printbibliography[heading=pubtype,title={\printinfo{\faFile*[regular]}{Journal Articles}},type=article]
%%
%%\divider
%%
%%\printbibliography[heading=pubtype,title={\printinfo{\faUsers}{Conference Proceedings}},type=inproceedings]

%% Switch to the right column. This will now automatically move to the second
%% page if the content is too long.
\switchcolumn

%%\cvsection{My Life Philosophy}
%%
%%\begin{quote}
%%``Something smart or heartfelt, preferably in one sentence.''
%%\end{quote}
%%
%%\cvsection{Most Proud of}
%%
%%\cvachievement{\faTrophy}{Fantastic Achievement}{and some details about it}
%%
%%\divider
%%
%%\cvachievement{\faHeartbeat}{Another achievement}{more details about it of course}
%%
%%\divider
%%
%%\cvachievement{\faHeartbeat}{Another achievement}{more details about it of course}

%%\cvsection{Strengths}
%%
%%\cvtag{motivation}
%%\cvtag{team spirit}\\
%%\cvtag{rigorous}

\cvsection{Compétences}
\cvtag{TCP/IP}
\cvtag{BGP}
\cvtag{OSPF}
\cvtag{BGP EVPN/VXLAN}
\cvtag{Traffic Engineering}
\cvtag{IPv6}
\cvtag{Debian}
\cvtag{Nftables}
\cvtag{Bird-FRR-Quagga}
\cvtag{YAML}
\cvtag{ipvsadm-HaProxy}
\cvtag{Vagrant-Virtuallab}

\divider\smallskip\divider
\cvtag{Cisco Nexus-ASR-Cat}
\cvtag{Juniper QFX-MX-SRX }
\cvtag{Nvidia-Cumulus/Mellanox }
\cvtag{Nortel}
\cvtag{Arista}
\cvtag{HP}
\cvtag{Dell}
\cvtag{F5 Viprion and GSLB}
\cvtag{Fortinet}
\cvtag{Cisco}

\divider\smallskip\divider
\cvtag{InfraAsCode}
\cvtag{Git}
\cvtag{Docker}
\cvtag{Jinja2}
\cvtag{Ansible}
\cvtag{Jerikan}
\cvtag{Bash, Python}
\cvtag{ZTP-ONIE}

%% Yeah I didn't spend too much time making all the
%% spacing consistent... sorry. Use \smallskip, \medskip,
%% \bigskip, \vspace etc to make adjustments.
\medskip

\cvsection{Formation}

\cvevent{ITESCIA, Ecole supérieure d’Informatique agréé par la CCI du Val d’Oise}{Master 2 Informatique, Réseaux et Télécoms, Management des SI. Alternance chez Orange et IBM}{2004-2007}{Cergy}

\cvevent{BTS Informatique de Gestion, option réseau – Lycée Turgot à Paris}{Stage Caisse des Dépôts et Consignations}{2002-2004}{Paris}

\cvevent{Certifications Cisco:}{CCNA, CCNP RS, CCNP SP : Module 642-887 SPCORE, CCDP 300-320 ARCH}{}{}
\cvevent{Certifications Juniper :}{JNCIA Junos, JNCIA DevOps}{}{}

\cvsection{Langues}

\cvskill{English - TOEIC 845 }{4}
\divider

\cvskill{Spanish}{2}
\divider

%%\cvsection{Referees}
%%
%%% \cvref{name}{email}{mailing address}
%%\cvref{Prof.\ Alpha Beta}{Institute}{a.beta@university.edu}
%%{Address Line 1\\Address line 2}
%%
%%\divider
%%
%%\cvref{Prof.\ Gamma Delta}{Institute}{g.delta@university.edu}
%%{Address Line 1\\Address line 2}

\end{paracol}

\end{document}
